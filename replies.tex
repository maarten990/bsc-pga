\documentclass[a4paper,11pt,twoside,openright]{article}

\usepackage{amsmath}
\usepackage{amssymb}
\usepackage[english]{babel}
\usepackage{color}
\usepackage{hyperref}
\usepackage{amsthm}
\usepackage{xspace}

\title{Replies to Leo's notes}
\author{\href{mailto:pkok@science.uva.nl}{Patrick de Kok}}

% markup for geometric terminology
\newcommand{\textgt}[1]{\textsf{#1}\xspace} 
% geometric terminology
\newcommand{\pBrush}{\textgt{Brush}}
\newcommand{\pBrushes}{\textgt{Brushes}}
\newcommand{\ebrush}{\textgt{brush}}
\newcommand{\ebrushes}{\textgt{brushes}}
\newcommand{\pLine}{\textgt{Line}}
\newcommand{\pLines}{\textgt{Lines}}
\newcommand{\eline}{\textgt{line}}
\newcommand{\elines}{\textgt{lines}}
\newcommand{\pPencil}{\textgt{Pencil}}
\newcommand{\pPencils}{\textgt{Pencils}}
\newcommand{\epencil}{\textgt{pencil}}
\newcommand{\epencils}{\textgt{pencils}}
\newcommand{\pPlane}{\textgt{Plane}}
\newcommand{\pPlanes}{\textgt{Planes}}
\newcommand{\eplane}{\textgt{plane}}
\newcommand{\eplanes}{\textgt{planes}}
\newcommand{\pPoint}{\textgt{Point}}
\newcommand{\pPoints}{\textgt{Points}}
\newcommand{\epoint}{\textgt{point}}
\newcommand{\epoints}{\textgt{points}}
\newcommand{\pRegulus}{\textgt{Regulus}}
\newcommand{\pReguli}{\textgt{Reguli}}
\newcommand{\pRegPen}{\textgt{Regulus Pencil}}
\newcommand{\pRegPens}{\textgt{Regulus Pencils}}

% Math-like symbols
\newcommand{\Rl}{\ensuremath{\mathbb{R}^{3,3}}}
\newcommand{\en}{\ensuremath{\mathbin{\mbox{and}}}}
\newcommand{\of}{\ensuremath{\mathbin{\mbox{or}}}}
\newcommand{\eql}{\ensuremath{\Leftrightarrow}}

\newtheorem{theorem}{Theorem}
\newtheorem{question}{Question}
\newtheorem{lemma}{Lemma}

\definecolor{nicered}{rgb}{0.6, 0, 0.24}
\definecolor{nicegreen}{rgb}{0.0, 0.5, 0.24}
\definecolor{niceblue}{rgb}{0, 0.4, 1}

\hypersetup{
  colorlinks=true,
  urlcolor=nicegreen,
  linkcolor=niceblue,
  citecolor=nicered,
}

\begin{document}
\maketitle

\section{2-blades}
\begin{question} 
Show that \emph{any} 2-blade of \Rl contains at least two \pLines.
\end{question}

This can be split up in 4 distinct cases.

\begin{enumerate}
\item \emph{A 2-blade of two intersecting \pLines $\ell_1 \wedge \ell_2$ contains at least two \pLines.}

See Leo's notes, 2.2, April 19.  Also: per definition.

\item \emph{A 2-blade of two skew \pLines $\ell_1 \wedge \ell_2$ contains at least two \pLines.}

See Leo's notes, 2.2, April 19.  Also: per definition.

\item \emph{A 2-blade of one \pLine and a non-\pLine $\ell \wedge k$ contains at least two \pLines.}

We are looking for \pLines.  As every null vector is a \pLine, we have the constraint $x^2 = 0$.

\[
\begin{array}{rlcl}
 & x \wedge (\ell \wedge k) = 0 &\en& x^2 = 0 \\
\eql& x = \alpha \ell + \beta k &\en& x^2 = 0 \\
\eql& (\alpha \ell + \beta k)^2 = 0 \\
\eql& \alpha^2 \ell^2 + \alpha \beta \left(\ell \cdot k\right) + \beta^2 k^2 = 0 \\
\eql& \alpha \beta \left(\ell \cdot k\right) + \beta^2 k^2 = 0 \\
\eql& \alpha \left(\ell \cdot k\right) = -\beta k^2 \\
\eql& \alpha = -\frac{\beta k^2}{\ell \cdot k} \en \beta \in \mathbb{R} &\of& \alpha \in \mathbb{R} \en \beta = -\frac{\alpha\left(\ell \cdot k\right)}{k^2}
\end{array}
\]

\item \emph{A 2-blade of two non-\pLines $k_1 \wedge k_2$ contains at least two \pLines.}

Again, $x$ should be a \pLine.  We have the same constraint $x^2 = 0$.

\[
\begin{array}{rlcl}
 & x \wedge (k_1 \wedge k_2) = 0 &\en& x^2 = 0 \\
\eql& x = \alpha k_1 + \beta k_2 &\en& x^2 = 0 \\
\eql& \left(\alpha k_1 + \beta k_2\right)^2 = 0 \\
\eql& \alpha^2 k_1^2 + \alpha \beta (k_1 \cdot k_2) + \beta^2 k_2^2 = 0 \\
\eql& \alpha \not= 0 \en \beta = \frac{\imath \left( \sqrt{3} \alpha k_1 \pm \imath \alpha k_1\right)}{2k_2} &\of& \beta \not= 0 \en \alpha = \frac{\imath \left( \sqrt{3} \beta k_2 \pm \imath \beta k_2\right)}{2k_1}
\end{array}
\]
This is according to Wolfram Alpha... I couldn't get further than this:

\[
\begin{array}{rlcl}
 & x \wedge (k_1 \wedge k_2) = 0 &\en& x^2 = 0 \\
\eql& x = \alpha k_1 + \beta k_2 &\en& x^2 = 0 \\
\eql& \left(\alpha k_1 + \beta k_2\right)^2 = 0 \\
\eql& \alpha^2 k_1^2 + \alpha \beta (k_1 \cdot k_2) + \beta^2 k_2^2 = 0 \\
\eql& \alpha \beta (k_1 \cdot k_2) = -\alpha^2 k_1^2 - \beta^2 k_2^2 \\
\eql& \alpha = - \frac{\alpha^2 k_1^2 + \beta^2 k_2^2}{\beta k_1 \cdot k_2} &\of& \beta = - \frac{\alpha^2 k_1^2 + \beta^2 k_2^2}{\alpha k_1 \cdot k_2} \\
\eql& \alpha^2 = -\frac{\alpha \beta (k_1 \cdot k_2) + \beta^2 k_2^2}{k_1^2} &\of& \beta^2 = -\frac{\alpha \beta (k_1 \cdot k_2) + \alpha^2 k_1^2}{k_2^2}
\end{array}
\]

I don't see how to get all $\alpha$'s or $\beta$'s on one side of the $=$...
\end{enumerate}
\qed

\begin{question}
Show that $\ell_1 \wedge \ell_2 \wedge \ell_3$, for relatively skew \pLines $\ell_1, \ell_2, \ell_3$ is a \pRegPen.
\end{question}
\end{document}
