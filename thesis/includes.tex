\usepackage{amsmath}
\usepackage{amssymb}
\usepackage{caption}
\usepackage{catoptions}
\usepackage{color}
\usepackage{etoolbox}
\usepackage{stmaryrd}
\usepackage{soul}
\usepackage{hyperref}

%% Math type specifying fonts
\newcommand{\V}[1]{\ensuremath{\mathbf{#1}}}
\newcommand{\s}[1]{\ensuremath{\mathcal{#1}}}

%% GA operators
\newcommand{\gp}{\ensuremath{\;}}
\newcommand{\gpi}{\ensuremath{\mathbin{/}}}
\newcommand{\lcont}{\ensuremath{\mathbin{\rfloor}}}
\newcommand{\rcont}{\ensuremath{\mathbin{\lfloor}}}
\newcommand{\dotp}{\ensuremath{\mathbin{\cdot}}}
\newcommand{\inverse}[1]{\ensuremath{#1^{-1}}}
\newcommand{\dual}[1]{\ensuremath{#1^\ast}}
\newcommand{\undual}[1]{\ensuremath{#1^{-\ast}}}
\newcommand{\edotp}{\ensuremath{\mathbin{\dotp_\mathbb{E}}}}
\newcommand{\elcont}{\ensuremath{\mathbin{\lcont_\mathbb{E}}}}
\newcommand{\ercont}{\ensuremath{\mathbin{\rcont_\mathbb{E}}}}
\newcommand{\edual}[1]{\ensuremath{#1 \lcont \inverse{\epseudo}}}
\newcommand{\eundual}[1]{\ensuremath{#1 \lcont \epseudo}}
\newcommand{\reverse}[1]{\ensuremath{\widetilde{#1}}}
%\newcommand{\edual}[1]{\ensuremath{#1^\bigstar}}
%\newcommand{\eundual}[1]{\ensuremath{#1^{-\bigstar}}}
\newcommand{\norm}[1]{\ensuremath{\left\|#1\right\|}}

%% LA operator
\newcommand{\transpose}{\ensuremath{\intercal}}
\newcommand{\m}[1]{\ensuremath{\left\llbracket #1 \right\rrbracket}}

\DeclareMathOperator{\pluckerid}{\Omega_q}
\DeclareMathOperator{\pluckerbilin}{\Omega}
\DeclareMathOperator{\Em}{Em}
\DeclareMathOperator{\grade}{\mathtt{grade}}
\DeclareMathOperator{\weight}{w}
\newcommand{\pdual}[1]{\ensuremath{#1^\natural}}

\newcommand{\heq}{\ensuremath{\mathbin{=_{H}}}} % Homogeneous equality

%% Pl\"ucker notation
\newcommand{\plucker}[2]{\ensuremath{\{#1 \mathbin{:} #2\}}}
%\newcommand{\pluckerpoint}[2]{\ensuremath{(#1 \mathbin{:} #2)}}
%\newcommand{\pluckerplane}[2]{\ensuremath{[#1 \mathbin{:} #2]}}

%% Math-like symbols
\newcommand{\reals}{\ensuremath{\mathbb{R}}}
\newcommand{\en}{\ensuremath{\mathbin{\mbox{and}}}}
\newcommand{\of}{\ensuremath{\mathbin{\mbox{or}}}}
\newcommand{\eql}{\ensuremath{\Leftrightarrow}}

%% Math constants and shorthands
\newcommand{\pline}{\ensuremath{\ell}}
\newcommand{\rline}{\ensuremath{\ell_o}}
\newcommand{\iline}{\ensuremath{\ell_\infty}}
\newcommand{\screw}{\ensuremath{s}}

\newcommand{\ez}{\ensuremath{e_0}}
\newcommand{\ee}{\ensuremath{\V{e}_1}}
\newcommand{\et}{\ensuremath{\V{e}_2}}
\newcommand{\ed}{\ensuremath{\V{e}_3}}
\newcommand{\eze}{\ensuremath{e_{01}}}
\newcommand{\ezt}{\ensuremath{e_{02}}}
\newcommand{\ezd}{\ensuremath{e_{03}}}
\newcommand{\etd}{\ensuremath{e_{23}}}
\newcommand{\ede}{\ensuremath{e_{31}}}
\newcommand{\eet}{\ensuremath{e_{12}}}
\newcommand{\epseudo}{\ensuremath{\V{I}_3}}
\newcommand{\hpseudo}{\ensuremath{\ez\V{I}_3}}
\newcommand{\ppseudo}{\ensuremath{I_6}}

\newcommand{\ap}{\ensuremath{a_{+}}}
\newcommand{\am}{\ensuremath{a_{-}}}
\newcommand{\bp}{\ensuremath{b_{+}}}
\newcommand{\bm}{\ensuremath{b_{-}}}
\newcommand{\cp}{\ensuremath{c_{+}}}
\newcommand{\cm}{\ensuremath{c_{-}}}

%% Text symbols
\newcommand{\newterm}{$^\dagger$}

\newcommand{\ega}{\texttt{e3ga}}
\newcommand{\pga}{\texttt{p3ga}}
\newcommand{\cga}{\texttt{c3ga}}
\newcommand{\cbga}{\texttt{c5ga}}
\newcommand{\iga}{\texttt{i2ga}}
\newcommand{\lga}{\texttt{l3ga}}

%% Setting color definitions
\definecolor{nicered}{rgb}{0.6, 0, 0.24}
\definecolor{nicegreen}{rgb}{0.0, 0.5, 0.24}
\definecolor{niceblue}{rgb}{0, 0.4, 1}
\definecolor{gray}{rgb}{0.2, 0.2, 0.2}
\definecolor{lightgray}{rgb}{0.4, 0.4, 0.4}

%% We want also \paragraph and \subparagraph to work!
\setcounter{secnumdepth}{5}

\captionsetup{
  margin=10pt,
  font=small,
  labelfont=bf,
  position=top,
}

\newcommand{\appendixfigure}[3][]{%
  \begin{figure}[#1]
    \caption{#3}
    \label{fig:#2}
    \begin{center}
      \fbox{
        \includegraphics[width=0.6\textwidth]{#2}
      }
    \end{center}
  \end{figure}
}

\newtoggle{draft}

\AtEndPreamble{
  \iftoggle{draft}{
    \hypersetup{
      colorlinks=true, 
      urlcolor=nicegreen,
      linkcolor=niceblue,
      citecolor=nicered,
    }
    \newcommand{\note}[1]{\footnote{#1}}
    \newcommand{\comment}[1]{\sethlcolor{red}{\bfseries\color{white} \hl{#1}}}
    \newcommand{\TODO}[1]{\sethlcolor{niceblue}{\bfseries\color{white} \hl{TODO: #1}}}
    \newcommand{\askLeo}[1]{\sethlcolor{yellow}{\bfseries \hl{Leo: #1}}}
    \newcommand{\LeosReply}[1]{\sethlcolor{green}{\bfseries \hl{Leo says: #1}}}
    \newcommand{\myReply}[1]{\sethlcolor{yellow}{\bfseries \hl{#1}}}
  }{
    \hypersetup{
      colorlinks=true,
      urlcolor=gray,
      linkcolor=gray,
      citecolor=lightgray,}
    \newcommand{\note}[1]{}
    \newcommand{\comment}[1]{}
    \newcommand{\TODO}[1]{}
    \newcommand{\askLeo}[1]{}
    \newcommand{\Leosays}[1]{}
    \newcommand{\LeosReply}[1]{}
    \newcommand{\myReply}[1]{}
    }}

\makeatletter
%% Provide \Autoref; the \autoref with a printed capital
\def\figureautorefname{figure}
\def\tableautorefname{table}
\def\partautorefname{part}
\def\appendixautorefname{appendix}
\def\equationautorefname{equation}
\def\AMSautorefname{equation}
\def\theoremautorefname{theorem}
\def\enumerationautorefname{case}
\def\Autoref#1{%
  \begingroup
  \edef\reserved@a{\cpttrimspaces{#1}}%
  \ifcsndefTF{r@#1}{%
    \xaftercsname{\expandafter\testreftype\@fourthoffive}
      {r@\reserved@a}.\\{#1}%
  }{%
    \ref{#1}%
  }%
  \endgroup
}
\def\testreftype#1.#2\\#3{%
  \ifcsndefTF{#1autorefname}{%
    \def\reserved@a##1##2\@nil{%
      \uppercase{\def\ref@name{##1}}%
      \csn@edef{#1autorefname}{\ref@name##2}%
      \autoref{#3}%
    }%
    \reserved@a#1\@nil
  }{%
    \autoref{#3}%
  }%
}

%% Commands for the title page.
% The title page needs the following data to be shown correctly:
% - \title
% - \subtitle (optional)
% - \studentid
% - \ects
% - \programme
% - \programmeaddress
% - \supervisor
% - \supervisoraddress
% - \finaldate or \date
\def\title#1{\gdef\@title{#1}}
\def\subtitle#1{\gdef\@subtitle{#1}}
\def\studentid#1{\gdef\@studentid{#1}}
\def\ects#1{\gdef\@ects{#1}}
\def\programme#1{\gdef\@programme{#1}}
\def\programmeaddress#1{\gdef\@programmeaddress{#1}}
\def\supervisor#1{\gdef\@supervisor{#1}}
\def\supervisoraddress#1{\gdef\@supervisoraddress{#1}}
\def\finaldate#1{\gdef\@finaldate{#1}}

\renewcommand{\maketitle}{
\begin{titlepage}
  \begin{center}

    \vspace{2.5cm}

    \begin{Huge}
      \@title
    \end{Huge}
    \rule{\linewidth}{0.1pt}

    \ifdef{\@subtitle}{
      \begin{large}
        \@subtitle
      \end{large}}{}

      \vspace{1.5cm}

      \@author\\
      \@studentid

      \vspace{1.5cm}

% [DO NOT CHANGE]
      Bachelor thesis\\
% [CHANGE] Whether your Bachelor thesis is 6 ECTS (regular) or 9 ECTS (Honours
% programme).
      Credits: \@ects{}EC

      \vspace{0.5cm}

% [DO NOT CHANGE] The name of the educational programme.
      \@programme

      \vspace{0.25cm}

% [DO NOT CHANGE] The addess of the educational programme.
      \@programmeaddress

      \vspace{4cm}

      \emph{Supervisor}\\
% [CHANGE] The name of your supervisor. Include the titles of your supervisor,
% as well as the initials for *all* of his/her first names.
      \@supervisor

      \vspace{0.25cm}

% [CHANGE] The address of the institute at which your supervisor is working.
% Be sure to include (1) institute (is appropriate), (2) faculty (if
% appropriate), (3) organisation name, (4) organisation address (2 lines).
      \@supervisoraddress

      \vspace{1.5cm}

% [CHANGE] The date at which you will finalize and submit your thesis.
      \ifdef{\@finaldate}{\@finaldate}{\@date}

    \end{center}
  \end{titlepage} 
}
\makeatother
