\section{Classifying and parameterizing elements of different geometric interpretations}
\label{ch:research}

Before an implementation of a visualization can be made for a model, the geometric aspects of the elements of the model must be investigated.  A full inventory of the geometric elements has been done~\cite[Chapter 3]{Pottmann}, but in different terms.  Those terms are traditional, but do not hint at the geometric concept.  Besides the traditional terms, geometric inspired terms will be used.  

The following sections will discuss all geometrically different blades of a certain grade, together with their duals.  Besides text, each part has an accompanying table that contains the following information on the discussed blades:
\begin{itemize}
  \item \texttt{g}: the grade of the blade.
  \item $B^2$: the square of the blade, if it is a scalar value.  It could be either greater than, equal to, less than or not equal to 0, respectively indicated with $> 0$, $= 0$, $< 0$ and $\not= 0$.
  \item $B^2$: the square of the blade.  The relation of this scalar value to zero is indicated in the table with $> 0$, $= 0$, $< 0$ and $\not= 0$.
  \item Form: the general formula.  A blade factor can be either a line or screw axis, denoted by $\pline$ and $\screw$.  In certain cases, the difference between a real and ideal line is made with $\rline$ and $\iline$.  Here it does not concern any algebraic difference.  The difference is based on the visualization.  In other cases, a small textual note is given to indicate that the factors coincide at a single point (``pt.\ int.'') or in a plane (``pl.\ int.''), intersect each other pairwise (``pw.\ int''), or no factors intersect at all (``skew'').
  \item Visualization: the picture of a typical blade of this type.
  \item Name: what the blade class is named.  Traditional names, if needed, have been referenced.  In case of non-geometrically inspired terms, a more geometric alternative is given.  These are marked with ``\newterm''.
\end{itemize}

\TODO{Add pictures!}

\subsection{Elements of grade 0 and 6}
\begin{table}
  \caption{An inventory of the blades of grade 0 and 6.}
  \label{tab:inv0}
  \begin{tabular}{|c|r|l|c|l|}
    \hline
    \multicolumn{1}{|c|}{\texttt{g}} & $B^2$ & \multicolumn{1}{|c|}{Form} & \multicolumn{1}{|c|}{Graphical representation} & \multicolumn{1}{|c|}{Name} \\ \hline
    \hline
    0 & $\in \reals$ & $1 = \dual{I_6}$ & & Scalar~\cite{TheBook} \\ \hline
    6 & $\in \reals$ & $I_6 = \dual{1}$ & & Pseudoscalar~\cite{TheBook} \\ \hline
  \end{tabular}
\end{table}

Scalars and pseudoscalars are interpreted the same as in other models for geometric algebra.  Scalars can be used to represent angles, weights, and many other scalar quantities.  Its dual, the pseudoscalar, may be utilized to denote the subspace volume, as well as for dualization.

\askLeo{Explain a bit more about pseudoscalar? Or scrap this whole section?}

\subsection{Elements of grade 1 and 5}
\begin{table}
  \caption{An inventory of the blades of grade 1 and 5.  
    \askLeo{Should I distinguish $\rline$ and $\iline$ in these tables?}}
  \label{tab:inv1}
  \begin{tabular}{|c|r|p{2.7cm}|p{2cm}|p{5cm}|}
    \hline
    \multicolumn{1}{|c|}{\texttt{g}} & $B^2$ & \multicolumn{1}{|c|}{Form} & \multicolumn{1}{|c|}{Visualization} & \multicolumn{1}{|c|}{Name} \\ \hline
    \hline
    1 & $= 0$ & $\rline$ & & Real line, path normal for uniform rotating motion\\ \hline
    1 & $= 0$ & $\iline$ & & Ideal line, path normal for uniform translating motion\\ \hline
    1 & $\not= 0$ & $\screw$ & & Screw axis \\ \hline
    5 & $= 0$ & $\dual{\rline}$ & & Singular linear complex~\cite{Pottmann}, uniform rotating motion \\ \hline
    5 & $= 0$ & $\dual{\iline}$ & & Singular linear complex~\cite{Pottmann}, uniform translating motion \\ \hline
    5 & $\not= 0$ & $\dual{\screw}$ & & Regular linear complex~\cite{Pottmann}, screw motion \\ \hline
  \end{tabular}
\end{table}

It is clear from \autoref{sec:plucker} that the null vectors of the Pl\"ucker model (those vectors $v$ satisfying the Pl\"ucker identity $\pluckerid(v) = 0$) are interpreted as lines.  Let $l = d_1 \eze + d_2 \ezt + d_3 \ezd + m_1 \etd + m_2 \ede + m_3 \eet$ be a null vector.  The line can be interpreted through the homogeneous model.  The direction of the line corresponds to $\Em(d_1 \eze + d_2 \ezt + d_3 \ezd) \lcont -\ez = (d_1 \ee + d_2 \et + d_3 \ed)$, while its moment is $\V{m} = \edual{\Em(m_1 \etd + m_2 \ede + m_3 \eet)} = m_1 \ee + m_2 \et + m_3 \ed$.

Ideal lines cannot be interpreted in this way.  The homogeneous model treats these purely Euclidean elements as bivectors.  In the Pl\"ucker model, there is no such thing as a purely Euclidean element, which leaves room for interpretation in a projective way.  These elements represent lines at infinity, or ideal lines.  One can visualize an ideal line $l_\infty = \beta_1 \etd + \beta_2 \ede + \beta_3 \eet$ as a horizon, a circle on the heavenly sphere in direction $\Em(l_\infty)$.  

The vectors that do not square to $0$ can be interpreted as the dual of a helical, or screw motion~\cite[Section 3.1.2]{Pottmann}.  A screw motion simultaneously translates and rotates an object.  The object is translated along the same line as around which the rotation is performed.  The motion can be fully characterized by this line $a$ and the pitch $p$, the amount of translation per rotation.  For the grade 5 screw motion $S = \dual{\plucker{\V{d}}{\V{m}}}$, its pitch and axis are given by $p = \V{d} \lcont \V{m} / \V{d}^2, a = \plucker{\V{d}}{\V{m} - p \V{d}}$~\cite[Theorem 3.1.9]{Pottmann}.  

From this, one can deduce what kind of objects the dual elements of real and ideal lines represent.  For an ideal line $l_\infty$ with $p = 0 \lcont \V{m} / 0^2 = \infty$ and $a = \plucker{0}{\V{m} - \infty 0} = l_\infty$, we see that it is a pure translation; even the slightest bit of rotation lets the object be translated.  Given a real line $l_o$, its pitch is $p = 0 / \V{d}^2 = 0$, because of the Pl\"ucker identity, seen in \autoref{eq:laplucker0}.  Its axis is $a = \plucker{\V{d}}{\V{m} - 0} = l_o$.  No matter how much of the rotation is applied, no translation will occur.  This shows that the real lines are the duals of pure rotational motions.

These results are presented summarized in \autoref{tab:inv1}. 


\subsection{Elements of grade 2 and 4}
\begin{table}
  \caption{An inventory of the blades of grade 2 and 4.}
  \label{tab:inv2}
  \begin{tabular}{|c|r|p{2.7cm}|p{2cm}|p{5cm}|}
    \hline
    \multicolumn{1}{|c|}{\texttt{g}} & $B^2$ & \multicolumn{1}{|c|}{Form} & \multicolumn{1}{|c|}{Graphical representation} & \multicolumn{1}{|c|}{Name} \\ \hline
    \hline
    2 & $= 0$ & $\pline \wedge \pline$, pt.\ int. & & Pencil of linear line complexes~\cite{Pottmann}, pencil of lines~\cite{Hongbo} \\ \hline
    2 & $> 0$ & $\pline \wedge \pline$, skew & & Skew line pair~\newterm \\ \hline
    2 & $\not= 0$ & $\pline \wedge \screw$ & & ??? \\ \hline
    2 & $< 0$ & $\screw \wedge \screw$ & & ??? \\ \hline
    4 & $= 0$ & $\dual{\left( \pline \wedge \pline \right)}$ & & Pencil of singular linear complexes~\cite{Pottmann}, dual pencil of lines \\ \hline
    4 & $> 0$ & $\dual{\left( \pline \wedge \pline \right)}$, skew & & Hyperbolic linear congruence~\cite{Pottmann}, dual skew line pair~\newterm \\ \hline
    4 & $\not= 0$ & $\dual{\left( \pline \wedge \screw \right)}$ & & ??? \\ \hline
    4 & $< 0$ & $\dual{\left( \screw \wedge \screw \right)}$ & & Elliptic linear congruence~\cite{Pottmann}, pencil of reguli~\newterm \\ \hline
  \end{tabular}
\end{table}

A two-blade $C = a \wedge b$ represents the set of linear combinations of $a$ and $b$:
\begin{equation*}
  \s{C} = \left\{ c \mid c \wedge C = 0 \right\} = \left\{ \lambda a + \mu b \mid \lambda, \mu \in \reals \right\} .
\end{equation*}
For the geometric interpretation of these blades, it should be clear what kind of objects are in this set.  On first inspection, this can be split in three cases, based on the number of null vectors among the factors.


When both $a$ and $b$ are null vectors, a linear combination $c = \lambda a + \mu b$ is a null vector if:
\begin{align*}
  c^2 &= (\lambda a + \mu b) \lcont (\lambda a + \mu b) \\
  &= \lambda^2 a^2 + 2 \lambda \mu a \lcont b + \mu^2 b^2 \\
  &= 2 \lambda \mu a \lcont b
\end{align*}

When $a$ and $b$ are coplanar, their inner product is zero (see \autoref{eq:coplanar}), and each linear combination is a null vector.  This kind of object is called a pencil of linear line complexes~\cite[Section 3.2.1]{Pottmann}, or, a bit shorter, a pencil of lines~\cite{Hongbo}.  The homogeneous point in which all lines intersect, is called its center~\cite{Hongbo}, and can be computed by \autoref{eq:intersect}.  The bivector direction containing the pencil can be computed by:

\begin{equation*}
  \V{D} = \Em(a) \lcont \ez \wedge \Em(b) \lcont \ez
\end{equation*}

\TODO{Skew lines}

\TODO{Line and screw}

\TODO{Screws}

 
\subsection{Elements of grade 3}
\begin{table}
  \caption{An inventory of the blades of grade 3.}
  \label{tab:inv3}
  \begin{tabular}{|c|r|p{2.7cm}|p{2cm}|p{5cm}|}
    \hline
    \multicolumn{1}{|c|}{\texttt{g}} & $B^2$ & \multicolumn{1}{|c|}{Form} & \multicolumn{1}{|c|}{Graphical representation} & \multicolumn{1}{|c|}{Name} \\ \hline
    \hline
    3 & $= 0$ & $\pline \wedge \pline \wedge \pline$, pt.\ int. & & Bundle of lines, point~\newterm \\ \hline
    3 & $\not= 0$ & $\pline \wedge \pline \wedge \pline$, pl.\ int. & & Field of lines, plane~\newterm \\ \hline
    3 & $\not= 0$ & $\pline \wedge \pline \wedge \pline$, skew & & Regulus~\cite{Hongbo} \\ \hline
    3 & $\not= 0$ & $\pline \wedge \pline \wedge \pline$, pw.\ int. & & Couple-wheel pencil~\cite{Hongbo}, line pencil pair~\newterm \\ \hline
  \end{tabular}
\end{table}


