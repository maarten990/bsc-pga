\section{Classifying and parameterizing elements of different geometric interpretations}
\label{ch:research}

Before an implementation of a visualization can be made for a model, the geometrical aspects of the elements of the model must be investigated.  A full inventory of the geometrical elements has been done~\cite[Chapter 3]{Pottmann}, but in different terms.  Those terms are traditional, but do not hint at the geometric concept.  Besides the traditional terms, geometric inspired terms will be used.  

The following sections will discuss all geometric different blades of a certain grade, together with their duals.  Besides text, each part will contain a table that contains the following column headers:
\begin{itemize}
  \item \texttt{g}: the grade of the blade.
  \item $B^2$: the square of the blade, if it is a scalar value.  It could be either greater than, equal to, less than or not equal to 0, respectively denoted with $> 0$, $= 0$, $< 0$ and $\not= 0$.
  \item Form: the general formula.  A blade factor can be either a line or screw axis, denoted by $\pline$ and $\screw$.  In certain cases, the difference between a real and ideal line is made with $\rline$ and $\iline$.  Here it does not concern any real algebraic difference.  The difference is based on the visualization.  In other cases, a small textual note is given to indicate that the factors coincide at a single point (``pt.\ int.'') or in a plane (``pl.\ int.''), intersect each other pairwise (``pw.\ int''), or no factors do intersect at all (``skew'').
  \item Visualization: the picture of a typical blade of this type.
  \item Name: what the blade class is named.  Traditional names, if needed, have been referenced.  In case of non-geometric inspired terms, a more geometric alternative is given.  These are marked with ``\newterm''.
\end{itemize}

\TODO{Add visualizations!}

\subsection{Elements of grade 0 and 6}
\begin{table}
  \caption{An inventory of the blades of grade 0 and 6.}
  \label{tab:inv0}
  \begin{tabular}{|c|r|l|c|l|}
    \hline
    \multicolumn{1}{|c|}{\texttt{g}} & $B^2$ & \multicolumn{1}{|c|}{Form} & \multicolumn{1}{|c|}{Graphical representation} & \multicolumn{1}{|c|}{Name} \\ \hline
    \hline
    0 & $\in \reals$ & $1 = \dual{I_6}$ & & Scalar~\cite{TheBook} \\ \hline
    6 & $\in \reals$ & $I_6 = \dual{1}$ & & Pseudoscalar~\cite{TheBook} \\ \hline
  \end{tabular}
\end{table}

Scalars and pseudoscalars are interpreted the same as in other models for geometric algebra.  Scalars can be used to represent angles, weights, and many other scalar quantities.  Its dual, the pseudoscalar, may be utilized to denote the subspace volume, as well as for dualization.

\askLeo{Explain a bit more about pseudoscalar? Or scrap this whole section?}

\subsection{Elements of grade 1 and 5}
\begin{table}
  \caption{An inventory of the blades of grade 1 and 5.}
  \label{tab:inv1}
  \begin{tabular}{|c|r|p{2.7cm}|p{2cm}|p{5cm}|}
    \hline
    \multicolumn{1}{|c|}{\texttt{g}} & $B^2$ & \multicolumn{1}{|c|}{Form} & \multicolumn{1}{|c|}{Visualization} & \multicolumn{1}{|c|}{Name} \\ \hline
    \hline
    1 & $= 0$ & $\rline$ & & Real line \\ \hline
    1 & $= 0$ & $\iline$ & & Ideal line \\ \hline
    1 & $\not= 0$ & $\screw$ & & Screw axis \\ \hline
    5 & $= 0$ & $\dual{\rline}$ & & Singular line complex~\cite{Pottmann} \\ \hline
    5 & $= 0$ & $\dual{\iline}$ & & Singular line complex~\cite{Pottmann} \\ \hline
    5 & $\not= 0$ & $\dual{\screw}$ & & Regular line complex~\cite{Pottmann}, regulus bundle~\newterm \\ \hline
  \end{tabular}
\end{table}

It is clear from \autoref{sec:plucker} that the null vectors of the Pl\"ucker model (those vectors $v$ which satisfy $v \lcont v = 0$) are interpreted as lines.  Let $\ell = \alpha_1 \eze + \alpha_2 \ezt + \alpha_3 \ezd + \beta_1 \etd + \beta_2 \ede + \beta_3 \eet$ be a null vector.  The line can be interpreted through the homogeneous model.  The direction of the line corresponds to $\Em(\alpha_1 \eze + \alpha_2 \ezt + \alpha_3 \ezd) \lcont -\ez = (\alpha_1 \ee + \alpha_2 \et + \alpha_3 \ed)$, while its moment is $\V{m} = \edual{\Em(\beta_1 \etd + \beta_2 \ede + \beta_3 \eet)} = \beta_1 \ee + \beta_2 \et + \beta_3 \ed$.

This only works with a nonzero direction.  The homogeneous model treats these Euclidean elements as bivectors.  In the Pl\"ucker model, there is no such thing as a purely Euclidean element, which leaves room for interpretation in a projective way.  These elements represent lines at infinity, or ideal lines.  An ideal line could be visualized as a circle which surrounds the space $\reals^3$ in the direction of $\Em(\ell) = \eundual{\V{m}}$.  Ideal lines do not need any special treatment.

\TODO{Interpret screws}

\TODO{Connect interpretation with grade 5 objects}



\subsection{Elements of grade 2 and 4}

\begin{table}
  \caption{An inventory of the blades of grade 2 and 4.}
  \label{tab:inv2}
  \begin{tabular}{|c|r|p{2.7cm}|p{2cm}|p{5cm}|}
    \hline
    \multicolumn{1}{|c|}{\texttt{g}} & $B^2$ & \multicolumn{1}{|c|}{Form} & \multicolumn{1}{|c|}{Graphical representation} & \multicolumn{1}{|c|}{Name} \\ \hline
    \hline
    2 & $= 0$ & $\pline \wedge \pline$, pt.\ int. & & Pencil of lines~\cite{Hongbo} \\ \hline
    2 & $> 0$ & $\pline \wedge \pline$, skew & & Skew line pair~\newterm \\ \hline
    2 & $\not= 0$ & $\pline \wedge \screw$ & & ??? \\ \hline
    2 & $< 0$ & $\screw \wedge \screw$ & & ??? \\ \hline
    4 & $= 0$ & $\dual{\left( \pline \wedge \pline \right)}$ & & ??? \\ \hline
    4 & $> 0$ & $\dual{\left( \pline \wedge \pline \right)}$, skew & & Hyperbolic linear congruence~\cite{Pottmann}, dual skew line pair~\newterm \\ \hline
    4 & $\not= 0$ & $\dual{\left( \pline \wedge \screw \right)}$ & & ??? \\ \hline
    4 & $< 0$ & $\dual{\left( \screw \wedge \screw \right)}$ & & Elliptic linear congruence~\cite{Pottmann}, pencil of reguli~\newterm \\ \hline
  \end{tabular}
\end{table}

\subsection{Elements of grade 3}
\begin{table}
  \caption{An inventory of the blades of grade 3.}
  \label{tab:inv3}
  \begin{tabular}{|c|r|p{2.7cm}|p{2cm}|p{5cm}|}
    \hline
    \multicolumn{1}{|c|}{\texttt{g}} & $B^2$ & \multicolumn{1}{|c|}{Form} & \multicolumn{1}{|c|}{Graphical representation} & \multicolumn{1}{|c|}{Name} \\ \hline
    \hline
    3 & $= 0$ & $\pline \wedge \pline \wedge \pline$, pt.\ int. & & Bundle of lines, point~\newterm \\ \hline
    3 & $\not= 0$ & $\pline \wedge \pline \wedge \pline$, pl.\ int. & & Field of lines, plane~\newterm \\ \hline
    3 & $\not= 0$ & $\pline \wedge \pline \wedge \pline$, skew & & Regulus~\cite{Hongbo} \\ \hline
    3 & $\not= 0$ & $\pline \wedge \pline \wedge \pline$, pw.\ int. & & Couple-wheel pencil~\cite{Hongbo}, line pencil pair~\newterm \\ \hline
  \end{tabular}
\end{table}
