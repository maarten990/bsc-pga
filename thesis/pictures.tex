\section{Visualizations}
\label{ch:visualization}

This appendix contains visualizations of the analyzed blades of \autoref{ch:research}, generated by the extended implementation of GAViewer described in \autoref{ch:implementation}.  The images are generated without changing the position of the viewport with respect to the origin.

For some objects, we show several differently parameterized versions to give the user an idea of how algebraic concepts are visualized.


%\subsection{Line}
%\label{sec:visline}
\appendixfigure[h]{linee01}{The real line $\eze$ with its orientation shown.  Compare the length of the arrows with those of \autoref{fig:linepair2} for the visualization of the weight.}

\appendixfigure[h]{linee12}{The ideal line $\eet$ with its orientation shown.  Its weight $w = 1$ is displayed by the radius of the circle.  Compare with \autoref{fig:screw2} for the visualization of its weight.}

\FloatBarrier
%\subsection{Screw axis}
%\label{sec:visscrew}
\appendixfigure{screw1}{The screw $(\eze+ 4 \etd) / \sqrt{1^2 + 4^2}$.  Its pitch $p = 4$ is shown by the length of the spiral.}

\appendixfigure{screw2}{The screw $4 (\ezd+ 4 \eet) / \sqrt{1^2 + 4^2}$.  Its weight $w = 2$ is displayed by the radius of the spiral.  Compare with \autoref{fig:linee12} for the visualization of the weight.}

\FloatBarrier
%\subsection{Pencil of linear line complexes}
%\label{sec:vispen}
\appendixfigure{pencil1}{The pencil of lines $\eze \wedge \ezt$ with no orientation shown.  Compare with \autoref{fig:pencil2} and \autoref{fig:pencil3} for the visualization of its weight and orientation.}

\appendixfigure{pencil2}{The pencil of lines $3 \eze \wedge \ezt$ with its orientation shown as vectors pointing.   Compare with \autoref{fig:pencil1} and \autoref{fig:pencil} for the visualization of its weight and orientation.}

\FloatBarrier

\appendixfigure{pencil3}{The pencil of lines $-\frac{1}{2} \eze \wedge \ezt$ with its orientation shown as vectors pointing.   Compare with \autoref{fig:pencil1} for the visualization of its weight and orientation.}

\appendixfigure{pencil4}{The pencil of lines $(\eze + \ezt) \wedge \eet$.  Compare with \autoref{fig:plane2} for the visualization of its weight and orientation.}

\FloatBarrier

\appendixfigure{ipencil}{The pencil of ideal lines $\etd \wedge \eet$ with its orientation shown. }

%\subsection{Hyperbolic linear line congruence}
%\label{sec:vishypercon}
\appendixfigure{linepair1}{The hyperbolic linear line congruence $\eze \wedge (\ezt + \etd)$.  Its orientation and weight are not shown.}

\FloatBarrier

\appendixfigure{linepair2}{The hyperbolic linear line congruence $-2 \eze \wedge (\ezt + \etd)$ with its weight and orientation shown.  Compare the length of the arrows with those of \autoref{fig:linee01} for the visualization of the weight.}


\appendixfigure{linepair3}{The hyperbolic linear line congruence $2 (\eze + \ezt) \wedge (\etd + \eet)$.  Compare with \autoref{fig:linepair2} for the visualization of its ideal line part.  Compare with \autoref{fig:linee12} for the visualization of the weight of its ideal line part.}

\FloatBarrier

%\subsection{Hyperbolic pencil of linear complexes}
%\label{sec:vishyperpen}
\appendixfigure{duallinepair1}{The hyperbolic pencil of linear complexes $\dual{(\eze \wedge (\etd-\ezt))}$.  For a good insight of this figure, also consider \autoref{fig:duallinepair3}.  Compare with \autoref{fig:duallinepair2} for the visualization of the orientation.}

\appendixfigure{duallinepair2}{The hyperbolic pencil of linear complexes $\dual{(\eze \wedge (\etd-\ezt))}$ with its orientation shown.  Compare with \autoref{fig:duallinepair1} for the visualization of the orientation.}

\FloatBarrier

\appendixfigure{duallinepair3}{The hyperbolic pencil of linear complexes $\dual{(\eze \wedge (\ezt+\etd))}$.  For a good insight of this figure, also consider \autoref{fig:duallinepair1}.}

\appendixfigure{duallinepair4}{The hyperbolic pencil of linear complexes $\dual{((\eze + \ezt) \wedge (\etd + \eet))}$.  Compare with \autoref{fig:linee01} and \autoref{fig:linepair3} for the visualization of a line without weight and orientation.  Compare with \autoref{fig:linepair3} for the color difference signifying grade, even though the drawing routines are the same.}

\FloatBarrier
%\subsection{Point}
%\label{sec:visrpoint}
\appendixfigure{point1}{The points $\eze \wedge (\ezt - \eet) \wedge (\ezd + \ede)$ (left) and $3 \eze \wedge (\ezt + \eet) \wedge (\ezd - \ede)$ (right).  Compare the two points; their radius is the visual representation of their weight.}

\appendixfigure{point2}{The ideal point $3 \eze \wedge \ede \wedge \eet$.  The distance between the two spheres is the visual representation of its weight.  Compare with \autoref{fig:point3} for the visualization of its orientation.}

\FloatBarrier

\appendixfigure{point3}{The ideal point $\eze \wedge \ede \wedge \eet$.  The vector from one sphere to the other is the visual representation of its orientation.  Compare with \autoref{fig:point3} for the visualization of its weight.}

%\subsection{Plane}
%\label{sec:visplane}

\appendixfigure{plane1}{The plane $\eze \wedge \ezd \wedge (\eze - \etd + 0.1 \ede + \ezd + \eet$.  Compare with \autoref{fig:plane2} for the visualization of its orientation and weight.}

\FloatBarrier

\appendixfigure{plane2}{The plane $-\eze \wedge \ezd \wedge (\eze - \etd + 0.1 \ede + \ezd + \eet$ with its weight and orientation visualized.  The little green bars pointing up are the visualization of its weight and orientation.}

\appendixfigure{plane3}{The ideal plane $\etd \wedge \ede \wedge \eet$ with its weight and orientation visualized.  Compare the radius of the sphere with \autoref{fig:plane4} for the visualization of the weight.}

\FloatBarrier

\appendixfigure{plane4}{The ideal plane $-4 \etd \wedge \ede \wedge \eet$ with its weight and orientation visualized.  The direction of the green bars indicate the plane's orientation. Compare the radius of the sphere with \autoref{fig:plane3} for the visualization of the weight.}

%\subsection{Couple-wheel pencil}
%\label{sec:vispenpair}

\appendixfigure{cwpencil}{The couple-wheel pencil $\eze \wedge (\ezt + \ezd) \wedge (\ezt + 2 \eet)$.  Note that the pencils are connected with each other through the common line.  The left pencil is tilted to the front.}
