\section{Conclusion}
\label{ch:conclusion}
This thesis presents a method for most blades in the space $\reals^{3,3}$ of the Pl\"ucker model.  This has been done through an embedding function $\Em$ of the Pl\"ucker model to the homogeneous model, provided by Li and Zhang~\cite{Hongbo}.  We have demonstrated that among the blades of $\reals^{3,3}$ are (real and ideal) lines, screw axes, pencils of linear line complexes, hyperbolic linear congruence, points, planes, couple-wheel pencils, and their duals.  Most of this is done through finding correspondences between the elements of a similar model for linear algebra, employed by Pottmann and Wallner~\cite{Pottmann} and those of our algebra.

We present a function to compute the weight and orientation of each element of our algebra.  We also show how necessary characteristics of the above stated geometric entities for visualization are computed.

GAViewer, a graphical calculator for various models of geometric algebra~\cite{GAViewer}, has been extended with the above knowledge.  Support for the algebraic model is added through code generated by Gaigen~\cite{Gaigen}, although support for an algebra used to represent 2D grey value images has been dropped.  GAViewer has casting operators, that work as embedding functions between other models and the Pl\"ucker model.  With the computed characteristics of the elements, GAViewer generates visualizations of the described geometric entities on user input without any noticeable delay.


But there is more work to be done in both our implementation as well as in developing an understanding of the Pl\"ucker model for geometric algebra.  We have only investigated the $(n>1)$-blades that can be generated by the outer product of null vectors.  Exploration of the other blades in this model for geometric algebra has to be done.  For instance, the parabolic and elliptic pencils of linear complexes and their congruences~\cite[Section 3.2.1]{Pottmann} are not demonstrated as members of the model, while they are in the similar model for linear algebra.  Our visualization of the elements of grade 5 is based on their duals' interpretation, but a more direct graphical representation might be beneficial for the user's understanding of the object.  The blades of grade 3 which are composed of three mutually skew lines need more analysis before they can be fully characterized.  

This thesis presents no transformations, although it is claimed~\cite{Hongbo} that the model is operational.  An introduction to these is given in an internal report~\cite{internal}.  One might investigate if it is useful for the user to have these visualized by GAViewer. 

We have shown that several 1- and 3-blades correspond geometrically to specific 1-, 2- and 3-blades of the homogeneous model.  A study of their relationship might reveal an embedding function of the Euclidean model in the Pl\"ucker model.

The model used is for the base space $\reals^3$, and it is easy to see a similar model for any real vector space of odd dimensionality.  In these other models, even more geometric entities might rise.  General relationships between these entities might be designed.

More interesting for its practical applications would be to develop a similar model for a real vector space of even dimensionality, such as $\reals^2$.  It might be realized within $\reals^{3,3}$ as the intersection with a plane, but this needs more research to see if this model has any practical significance.
