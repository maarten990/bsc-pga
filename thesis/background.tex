\section{Projective geometry and geometric algebra}
\label{ch:background}

\subsection{Geometric algebra}
\label{sec:intro-ga}
\comment{Paraphrase Moos's work?}

\comment{Include the next notation explanation:}
\begin{itemize}
  \item Bold for Euclidean
  \item Capital for multivectors
  \item Lowercase for vectors/1-blades
  \item Greek for scalars/0-blades
  \item $\RL$
  \item Geometric product: $A \gp B$ $\leftarrow$ small space
  \item Geometric division $A \gpi B$ or too trivial?
  \item Geometric division $A \mathbin{/} B$ or too trivial?
  \item Outer product $\wedge$
  \item Left contraction $\lcont$
  \item Dot product $\dotp$
  \item Dualization $\dual{A}$, undualization $\undual{A}$
  \item Euclidean (un-)dualization $\edual{\V{A}}$, $\unedual{\V{A}}$
\end{itemize}

\subsection{Pl\"ucker model}
\label{sec:hongbo}
\comment{This part will be paraphrased from \cite{TheBook} and \cite{Hongbo}.}

In the homogeneous model, the line through points $p = \ez + \V{p}$ and $q = \ez + \V{q}$ is represented as 
\begin{equation}
  L = p \wedge q = \ez \wedge (\V{q} - \V{p}) + \V{p} \wedge \V{q} .
\end{equation}

The direction and moment of a line are easily recognised in this expression.  The direction $\V{d} = \V{p} - \V{q}$ is encoded in the first factor as $\ez \wedge -\V{d} = \V{d} \wedge \ez = \V{d} \ez$.  
The moment $\V{m} = \V{p} \times \V{q} = \edual{(\V{p} \wedge \V{q})}$ can be found in the second term as $\unedual{\V{m}} = \unedual{(\edual{(\V{p} \wedge \V{q})})} = \V{p} \wedge \V{q}$, which results in another general formula for lines:
\begin{equation}
  L = \V{d} \ez + \unedual{\V{m}}.
\end{equation}

Within classical literature of linear algebra, the same object is written as $-\plucker{\V{d}}{\V{m}}$, using \emph{Pl\"ucker coordinates} to denote a line as a 6D vector.  
